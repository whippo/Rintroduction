\documentclass{article}

\usepackage{Sweave}
\begin{document}
\Sconcordance{concordance:Handbook_Sweave.tex:Handbook_Sweave.Rnw:%
1 2 1 1 0 538 1}


\section{Introduction to R, libraries, vectors, data frames, and functions}

  R is a computer language that is open-source and free, and in the last decade it has been widely adopted by scientists as a way to perform statistical analyses ranging from a basic t-test, to incredibly complex multivariate Bayesian modeling. One of the great advantages of R is that, unlike some other expensive statistical software, it is not a 'black box' that simply swallows your data and spits out an answer. It is a transparent coding procedure that can be shared and replicated, leaving a 'paper trail' in the form of code that allows you to see exactly how you get from dataset to statistical result. 

  The purpose of this workshop is to familiarize you with the basic format and capabilities of R, and point you in the right direction for future self-study\footnote{A note: some of the data and text of this workshop is derived from a Biostatistics Course taught by Dr. Edd Hammill at the Bamfield Marine Sciences Centre in 2012.}. This is not an exhaustive R course, and undoubtedly you will still have many questions at the end. However, it is my hope that learning some of the syntax, frequently used commands, and shortcuts will allow you to better determine what type of approach to coding in R will work best for you. It really is true that if you give 20 people a problem to work out in R, you'll end up with 20 different solutions. 

\begin{itemize}
\item Advantages to using R: \textit{free, flexible, well-maintained, transparent, easy to share, used in many fields}
\item Disadvantages to using R: \textit{steep initial learning curve, intimidating, unyieldingly precise}
\end{itemize}

\subsection{Preparation}

  Before the workshop please download and install R and R Studio for your computer. Both are available for Windows, Mac, and Linux, and some OS even come with R preinstalled. 
  
\bigskip
R is the architecture and language framework that does all the heavy lifting for your analyses. You can download it here:

\bigskip
	https://cran.r-project.org/
	
\bigskip
For Windows follow the link and instructions for Windows. For Mac, follow the Mac link and download the file with the '.pkg' suffix on the left side to install. For Linux (Ubuntu) you should already have R preinstalled, if not, download it from the Ubuntu Software Center.

\bigskip
R Studio is a graphical user interface (GUI) that allows you to easily interact with, run, and save R coding scripts. You can download it here: 

\bigskip
	https://www.rstudio.com/products/rstudio/download/\#download

\bigskip
Just choose your OS and it should install with no problems. Congratulations!! You’ve just acquired the most powerful and versatile stats package in the world. 

\subsection{Goals of this Workshop}

By the end of this workshop, you should have a basic understanding of the following procedures and techniques:

\begin{itemize}
    \item loading datasets into R
    \item downloading and installing statistical/manipulative/graphical packages for R
    \item producing elegant(ish), well-annotated code
    \item manipulating datasets in R
    \item identifying the most parsimonious ways to enter data into spreadsheets for use in R
    \item performing basic statistical tests of data including t-tests and ANOVAs
    \item making simple visualization of data and results
    \item understanding what version control is and why it's important
    \item know where to look for help when you're stuck
\end{itemize}

\subsection{R Interface Basics}

Open R Studio.
Your first reaction will most likely be overwhelming disappointment and a sense of “is that it?” when you’re presented with…

R version 3.4.4 (2018-03-15) -- "Someone to Lean On"
Copyright (C) 2018 The R Foundation for Statistical Computing
Platform: x86\_64-pc-linux-gnu (64-bit)

R is free software and comes with ABSOLUTELY NO WARRANTY.
You are welcome to redistribute it under certain conditions.
Type 'license()' or 'licence()' for distribution details.

  Natural language support but running in an English locale

R is a collaborative project with many contributors.
Type 'contributors()' for more information and
'citation()' on how to cite R or R packages in publications.

Type 'demo()' for some demos, 'help()' for on-line help, or
'help.start()' for an HTML browser interface to help.
Type 'q()' to quit R.

>



Yes, that is it. You are currently looking at the “R console”. R is a programming package, you have to “tell it what to do”. Graphical user interfaces (GUIs) are available and are much more like the drop-down menu, point-and-click interface you’re used to, but they tend to be cumbersome and a pain. R works more like a very powerful calculator, you have to actually program the commands into it. As part of this, R has it’s own language, and during the rest of this course we will learn aspects of that language. 



Another note from Ross: DOWNLOAD THE TIDYVERSE!
https://www.tidyverse.org/
(the link above is primarily informative, you can download the packges in R by entering the command: install.packages('tidyverse')
If this command throws you an error, download the most useful packages one at a time:
install.packages('dplyr')
install.packages('tidyr')
install.packages('ggplot2')


Inputting commands into R
There are 2 ways to input into R, these are….
    1. Type things directly into the console, try this now, type “2*4” and press return, or anything else you fancy. You’ll see it works a bit like a calculator. 
    2. Copy and paste from another window. This is a much better way of using R, and the habit I want you to get into. It means you have a record of everything you’ve done that you can easily edit. 

In R, click “File”, and “New Document”. You’ll now be presented with a basic text editor. You can type code into this editor, and easily copy it across to the R Console by pressing “Command + Return” in a Mac, or “Control + R” in a PC. 

Just try it with another simple command e.g. 4 times 5, then press enter, you should get…

> 4*5
[1] 20

Believe it or not you just wrote and ran a program. That text file you just wrote in is now technically a piece of software known as a “script”. This one contains a very simple piece of code. Congratulations, you’ve overcome the first step. 



Functions and naming things
Not to take away from your achievement, but inputting calculations one at a time into R isn’t really that useful. However, we can name things in R, and then use them again later. For example, say we want a vector of numbers between 1 and 10, and will call it “x.values”. R doesn’t like spaces, so we would input either..

> x.values<-c(1,2,3,4,5,6,7,8,9,10)

Or…

> x.values<-c(1:10)

The “<-“ in R basically means “is”, it’s naming something in a way that R can understand and find later. The “c” means “is a list of..”, and is a very useful command. You can name vectors, data frames, and functions (more on these later). Now type “x.values” and press enter, you should get

> x.values
[1]  1  2  3  4  5  6  7  8  9 10

The [1] at the beginning is just telling us that we’re starting from the beginning (number 1) of the “x.values”. If the list goes over one line, R will give you another number in brackets telling what number the start of that line is.

Naming things in this way is useful as you can use them for other calculations. For example, we could also produce the squares of our x.values by typing the following, the “\^” symbol means “to the power of”

> y.values<-x.values\^2
> y.values
[1]   1   4   9  16  25  36  49  64  81 100

We now have 2 vectors that R can understand. Of course what we named them wasn’t important, we specified that ourselves. These names are just a way for us to communicate with R. 

If we want to plot the two vectors against each other, we can use the inbuilt function plot().  There’s a huge array of functions in R, in each case the function is carried out on whatever is in the parenthesis. 

> plot(x.values,y.values)

This will produce a very basic plot of our 2 vectors. Within the plot function are many customisable elements called “arguments”, to find them, we can start by using the incredibly useful “help” function. 

> help(plot)

There’s all sorts to play with in there, don’t worry if you don’t understand it all, but as an example, lets make our plot a big red dashed line…

> plot(x.values,y.values,col="red",type="l",lty=3,lwd=3)

If we wanted to combine our 2 vectors into a data frame (similar to an excel spreadsheet in format) we simply use the “data.frame” command.

> x.and.y<-data.frame(x.values,y.values)
> x.and.y
     x.values y.values
1         1        1
2         2        4
3         3        9
4         4       16
5         5       25
6         6       36
7         7       49
8         8       64
9         9       81
10       10      100

These data frames are how we will be using most of our own data in R. 

As well as the functions built within R, we can make our own. Say for example we wanted to make a function that worked out the square root of something, and then added 2. We would program it, and have to name it something…

> my.function<-function(X){sqrt(X)+2}

The parenthesis () are the thing to which the function is applied to, the curly brackets {} are the actual function. We can then apply this to a number…

> my.function(9)
[1] 5

Or a vector, such as our “x.values” from before.

> my.function(x.values)
 [1] 3.000000 3.414214 3.732051 4.000000 4.236068 
 [6] 4.449490 4.645751 4.828427 5.000000 5.162278

You may need to produce your own functions later down the line in order to perform transformations in data. If you don’t ever need to make one, it’s worth knowing how they’re used as it will give you a better understanding of how they come to be in R. 

Task 1… Combining some functions to do a basic test

The function ”rnorm” produces a normal distribution, we can use it to make up simulate some data.  For example if we wanted to simulate 10 data points, with a mean of 8 and a standard deviation of 3, we would input

> data.1<-rnorm(n=10,mean=8,sd=3)
> data.1
[1] 11.694046 11.956468 11.174013  6.000238  1.365149 
[5] 12.945742  6.089448 10.645392 11.408550 12.508761

Don’t worry if you don’t get the exact same numbers, R is simulating the data and so it will be different each time.

Now, a little test. I want you to produce 2 vectors with rnorm, call one “data.1” and the other “data.2”. The first will be 20 numbers long, have a mean of 15 and a standard deviation of 3. The second will be 15 numbers long, have a mean of 9, and a standard deviation of 2. Have look at your 2 vectors, calculate the means and standard deviations using the functions “mean()” and “sd()”.

Running a t-test
Using “t.test”, see if there is a difference between your vectors (p < 0.05)? If you have trouble with “t.test”, try the help function then ask me. The t.test you are running is asking if there is a difference between the 2 groups. 

The t-test tests the null hypothesis (H0) that there is no difference between the groups

The Alternative hypothesis (Ha) is that there is a difference between the groups. 

Hints and tips for a better R life.

When producing R scripts, most people like “to annotate” their scripts to keep track of what they’re doing. R doesn’t read anything in a line after a “\#” symbol, so you can use this feature to add in notes for future reference.

E.g. in your script could write

x.values<-c(2:15) \#\# Making a vector of x values

\#\# now to make a list of y.values, these will be the 
\#\# x.value\^3

y.values<-x.values\^3

\#\# Now lets plot that..

plot(x.values,y.values)

You can copy all of that into R in one go, and R will ignore all the stuff behind the \#’s. This is very useful later on, and a good habit to get into. 




Bringing a spreadsheet from excel to R
Despite how wonderful R is for analysis, the fact remains that actually inputting data into a computer is best done in excel. This is the way in which most of you will be used to inputting data, and will do so in the future.  Inputting into R is often a real hurdle and fraught with difficulty for a lot of people. There are multiple ways to achieve it, I’m teaching you the one I most often use. The key point is that the syntax, and file name have to be exactly right.  There are a number of other extra things that can also go wrong, don’t worry if it doesn’t work immediately. We’ll go through it step by step. 

We’re going to use a small data set I collected in Costa Rica. The place we were staying in was full of scorpions, I decided to see if they were bigger in the kitchen or the bedroom. Each time we caught one in either room, we measured it then threw it the hell outside. The lengths are in “scorpion lengths.xls”. We’re going to import this data, obtain descriptive statistics (mean, standard deviation, standard error) then do a t.test again. 

Important point 1 - In excel, save the spreadsheet as a .csv file. Excel asks you if you want to do this in the “save as” menu. 

The procedure is now different if you’re using a PC or a Mac, We’ll do PCs first. 

PC procedure

In excel, save the file into a folder where you can easily find it, and with a name you’ll recognise, e.g. “R stuff” in Documents, you will save all future data here.

In R, go to “File”, click “Change directory”, find the folder you just saved the file in, and choose this as your directory. 

Now input the following line of code

> scorpion.data<-read.csv(“scorpion lengths.csv”,header=TRUE)

“read.csv” is a function like we’ve used before, the “header=TRUE” argument just tells R that the top row is the column names.  What we’ve done is brought the data in using the read.csv function, and named it scorpion.data. Type its name to have a look at it.

> scorpion.data
   bedroom kitchen
1      74      46
2      67      41
3      85      65
4      56      45
5      75      37
6      71      NA
7      67      NA


Mac procedure
In a Mac the key initial step is to save the file in a folder in your home directory, detailed instructions are below. 

In excel, open the scorpion lengths.xls file. Click “save as”, then…..
    1. On the left of the screen, under “PLACES” click your home directory (the little house). 
    2. Click “New Folder”, call it something like “r stuff”, you will save all future data in here. 
    3. Change the file format to “Comma Separated Values (.csv) 
    4. Save the file

In R, input the following line of code.

> scorpion.data<-read.csv("~/r stuff/scorpion lengths.csv",header=TRUE)

Each of the “/”s means “open this”, and what you just typed in is known as a “file path”. Now just have a look at it to make sure it’s there.

> scorpion.data
  bedroom kitchen
1      74      46
2      67      41
3      85      65
4      56      45
5      75      37
6      71      NA
7      67      NA

Everyone back together again! Now run descriptive statistics and do the t-test

The problem is the things we want to test this time aren’t separate vectors as they were before, but columns in a data frame. We need to find away to tell R to look within the data frame. There are two basic ways to do this, each with advantages and disadvantages.

    1. Telling R directly to look in the data frame

We can do this using “$”, this symbol means tells R to look inside. For example, if we wanted R to show us the lengths of the scorpion from the bedroom, we can type
$
> scorpion.data$bedroom
[1] 74 67 85 56 75 71 67
$
That’s telling R that we’re interested in the “bedroom” column, which is inside the “scorpion.data” data frame. You can the use these vectors as we did before for things like “plot” and all the rest of it.

You can now run descriptive statistics on these data by using commands such as 

mean() \# Calculates the mean of whatever is in the
	  \# brackets

sd() \# Calculates the standard deviation of whatever’s in
     \# the brackets

length() \# tells you the “length” of whatever’s in the
         \# brackets, in this case the number of values we
         \# have (n)

Task! Use the “$” method to calculate means and standard deviations of the lengths of scorpions from both locations. Then, do a t.test comparing the lengths of the scorpions. If you manage this easily, write a function to calculate the standard error, and run it on the scorpion data. 
$
Generally, I prefer the “\$” method, although it requires more typing it avoids attaching data frames together, which can cause problems down the line.

    2. Using the “attach” function

Rather than selecting columns, we can attach the data frame together, letting R know that we’re interested in using for now. 

> attach(scorpion.data)  \#\# Attaching the frame together

We can now run all the same commands as before, but don’t need to use the “$”
$
> t.test(kitchen,bedroom) 
	Welch Two Sample t-test

data:  kitchen and bedroom 
t = -4.0651, df = 7.65, p-value = 0.003964

The issue with this technique is that if we don’t “detach” the data frame afterwards, it languishes around inside R and gets in the way. If we later try to refer to something with the same name as one of our columns, R gets confused and can start to compare the wrong things. This can lead to all kinds of frustration. So always detach when you’ve finished with something.

> detach(scorpion.data) \#\# Make sure to do this!

When you’ve got to this stage, you’ve made some serious progress. You’ve overcome the first two big challenges of R, talking to it, and getting it to bring in data. At some point down the line, you will struggle to get R to bring in data frames. It happens to everyone, normally when someone’s watching you type. Don’t worry, just work through it slowly, check the command and file path syntax, and it will all fall into place. 

Before/after data and paired t-tests
A wonderful feature of t-tests is their ability to deal with paired data, for example looking at before/after the application of a treatment, or looking at the same locations at different times. 

For example, saw we wanted to ask whether the population density of Daphnia (a freshwater zooplankton species) differed between the summer and the winter. We could select 10 ponds, take a 1000ml sample of pond water, and count the number of individuals it contained. We may end up with the following data…

	daphnia.csv

pond summer winter
      1     30     13
      2     54     45
      3     63     51
      4     51     43
      5     32     21
      6     36     30
      7     42     35
      8     48     37
      9     39     31
     10     61     51

We now have repeated measures of our 10 different locations, and could do a t-test as before

t.test(daphnia$summer, daphnia$winter)

	Welch Two Sample t-test

data:  daphnia$summer and daphnia$winter 
t = 1.8354, df = 17.921, p-value = 0.0831

From this is appears as though there’s no difference in Daphnia population sizes between the seasons. However, could we be missing something? Our data here are not technically a random sample, as we’re sampling from the same ponds multiple times, and we’d therefore expect two samples from the same pond to be more similar than two samples from different ponds. Ponds may differ in many important ways, for example some may contain fish that can consume Daphnia, some may contain more food for Daphnia than others. As a consequence, the differences between the ponds are so large, that the variance of the two samples becomes very large. The data does have a natural pairing however (each site was sampled twice), and we can use this pairing as a sort of natural control. What if, instead of asking if there was a difference between the samples, we took the difference between the summer and winter value for each ponds, and see if the overall difference is significantly different from 0. If it is different from 0, then we’re essentially saying there was a difference between the summer and winter sample. 

Fortunately, we don’t need to do this by hand, as R can do the whole thing automatically. All we have to do is set the “paired” argument in the t-test to TRUE.

t.test(daphnia$summer, daphnia$winter,paired=TRUE)

	Paired t-test

data:  daphnia$summer and daphnia$winter 
t = 9.9611, df = 9, p-value = 3.696e-06

You can see that now the result is significant. We don’t have as many degrees of freedom as before, as technically we’ve only got half as many data points (1 list of differences, instead of 2 lists of the actual values). Remember though, this technique can only be used when there is a natural pairing to the data. 

“Long form” data in R
So far we’ve just looked at comparing 2 lists of data. It is however possible to use a slightly different method. Whenever we’re performing a statistical test, we have at least 2 variables, a response variable (the thing we’re interested in) and a descriptive variable (the thing we want to use to group the data). For example, say we wanted to see if there was a difference in height between men and women, “height” would be our response variable, and “sex” would be our descriptive variable. If we had the data in a data frame named “heights”, which contained 2 columns; “height” – the height in cm, and “sex” – either male or female, overall the data would look like…

height    sex
1     186   male
2     179   male
3     168   male
4     190   male
5     183   male
6     175   male
7     169   male
8     176   male
9     162 female
10    159 female
11    162 female
12    143 female
13    142 female
14    157 female
15    152 female

we could perform the following test….

t.test(height~sex,data=height)

This would generate results identical in form to the ones we saw before..

Welch Two Sample t-test

data:  height by sex 
t = -5.7667, df = 12.377, p-value = 7.909e-05

The “~” inside the t.test literally translates as “by”, so with the command line we’re saying “do a t.test looking at height differentiated by sex”. This method of using the “~” (called a “tilda”) is very common in R, and we shall be using it for the majority of the remaining tests. It is very important as it allows you to perform tests using multiple descriptive variables, and have multiple levels within a variable. But more on that later. 

Now try using the “~” method for yourself. Import the dataset “scorpion long” into R, and run the t.test, check you got the same result as when you ran the t.test before. You should do, you’re using the same data, it’s just organised differently. 


T-test notes
If you’re interested in how the t-test works, extensive information is available online. Essentially, it uses the following formula



Where the x’s are the means of the 2 samples, sx1x2 is the pooled standard deviation, and n is the total number of samples. We also need to know the number of degrees of freedom (the sample size n – 1). We can then look up our value of t with the appropriate number of degrees of freedom in t-tables, and if it is greater than the critical value, we have a significant difference. For example, say we had 20 samples, and found a t value of 5.60, we would look up the critical value (p < 0.05) of t on 19 degrees of freedom (2.093). As ours is bigger, we have a significant difference. We would then report it in the following manner (t(19) = 5.60, p < 0.05). Of course, R does all this for us, and we just need to read the t-value, the degrees of freedom, and the p-value.  

Task! 
Now you’ve managed to bring in data sets and communicate with R, have a go at doing the following..

    1. Bring in the data set “bromeliad.xls” (consult the instructions above for all the steps!)
    2. Make a plot with max.vol on the x-axis, and mosquitoes on the y.
    3. Perform a t-test to see if the number of mosquitoes differed between plant species (the “species” column in the dataset). 

Introduction to ANOVA

We previously looked at using a t-test to ask whether two different groups of data are significantly different from each other. We tested the null hypothesis of no difference between groups. In our case we had scorpion lengths from two different rooms on a research station. However, often we have data from more than two different groups (for example, if we’d found scorpions in the kitchen, bedroom AND bathroom, heaven forbid). In this situation we may want to ask if any of them differ from each other, for this we need a slightly different test, called “the Analysis of Variance”, or ANOVA. 

ANOVA is a very powerful, useful and versatile statistical test. It is at the heart of most of the statistics currently used, including (but not limited to) regression, linear modelling, non-linear modelling, mixed effects, MANOVA, analysis of covariance. As a result, you really want to understand it and know how to use it. 

The first thing to understand, is what an ANOVA actually tests…

Null hypothesis (H0) = There are no differences between the groups

Alternative hypothesis (HA) = at least one group is different from one other. 

It is important to understand that the alternative hypothesis is not saying ALL groups are different from ALL others, just that at least one is different from one other. For example, if we had 3 groups; A, B, and C the alternative hypothesis (HA) is…. A≠B≠C, 
    or A≠B B=C but A≠C, 
    or A≠B A=C C≠B
    or A≠B A≠C but B=C
    etc…….

In the t-test, the test relied on a 

Formally, when we run an ANOVA, our groups are known as our treatment, and each of our groups are formally known as levels of the treatment. The actual ANOVA test calculates an f-statistic, and then looks at whether this value is greater than the critical value of f with the correct treatment degrees of freedom (k-1), and our error degrees of freedom ((n-k)-1). Of course, being wonderful, R does all this for you and you just have to know where each value is, and what it means.

For example, lets take another look at the bromeliad data. 

data<-read.csv("~/bromeliads.csv",header=TRUE)
data   \#\#\# Taking a look at it…


leaf.number   diameter   max.vol well.vol mosquitoes species    location
10    58.5000000     450  45.00000         12 guzmania     pitilla
10    63.5000000     900  90.00000         67 guzmania Monte.verde
12    87.0000000     200  16.66667          3 guzmania     pitilla
13   113.5000000     875  67.30769         16 guzmania     pitilla
13    93.0000000     500  38.46154         35 guzmania Monte.verde
14    60.0000000     450  32.14286         24 guzmania Monte.verde
15    72.0000000     200  13.33333         15 guzmania     pitilla
15    80.0000000    1250  83.33333        113  verasia    De.salva
16   112.5000000    1250  78.12500         95  verasia Monte.verde
16    96.0000000    1000  62.50000        100  verasia    De.salva
16    84.0000000    1000  62.50000        110  verasia    De.salva
18    87.5000000    1500  83.33333        102  verasia    De.salva
19    74.0000000     750  39.47368         12 guzmania     pitilla
20    89.5000000    1140  57.00000         32 guzmania Monte.verde
20     0.5288462     500  25.00000         36 guzmania Monte.verde
21    95.5000000     500  23.80952          3 guzmania     pitilla
21    80.0000000     300  14.28571         15 guzmania     pitilla
21    85.0000000     625  29.76190         88  verasia Monte.verde
21   135.5000000    3000 142.85714        148  verasia     pitilla
22   107.5000000    1500  68.18182        105  verasia    De.salva
25    95.0000000    1250  50.00000         83  verasia Monte.verde
25    91.5000000    3000 120.00000        152  verasia     pitilla
26   115.0000000    1100  42.30769        121  verasia    De.salva
27    84.5000000     750  27.77778         34 guzmania    De.salva
27    98.0000000     750  27.77778         38 guzmania Monte.verde
27    87.0000000    1125  41.66667        120  verasia    De.salva
34    69.0000000    1500  44.11765         52 guzmania Monte.verde
36   102.5000000    1250  34.72222        100  verasia    De.salva
40    94.5000000     500  12.50000          6 guzmania     pitilla
42   107.5000000    1250  29.76190        102  verasia    De.salva

Last time, we used the column “species” to run a t-test, but say for example we wanted to see if the number of mosquitoes differed between the 3 locations, “De salva”, “pitilla”, and “Monte verde”. We now have to run an ANOVA as we have more than two groups, we do this using the “aov” command in R. 

ANOVAs run slightly differently than t-tests, and we have to actually name the test something, we’ll call it “anova.1”. As the data is in long form we have to use the “~”, because we’ve not attached it we also use the “$”. Remember, R won’t read anything after the “\#”
$
anova.1<-aov(data$mosquitoes~data$location) \#\# test code

When you run the test, it doesn’t automatically give you the output, this is because you’ve just produced the test within R, to look at the results, you need to use the “summary” command..

summary(anova.1)

This should then give you the following ANOVA table…



              Df Sum Sq  Mean Sq  F value   Pr(>F)   
data$location  2  20923   10462    6.55    0.0048 **
Residuals     27  43124    1597                  
---
Signif. codes:  0 ‘***’ 0.001 ‘**’ 0.01 ‘*’ 0.05 ‘.’ 0.1 ‘ ’ 1 
$
This contains everything you need to report your result, R has calculated the f-statistic, the p-value, and all the degrees of freedom (Df) for you, and even told you the significance level. 

In our case, “data$location” was our treatment, and you can see from the degrees of freedom column the treatment df=2 (remember, treatment degrees of freedom equal number of levels minus one). It has given us our error degrees of freedom (Although R calls these “Residuals”, just to be difficult). So how do we report this, bearing in mind what our original question was; are there differences in mosquito abundance between the locations?
$
To formally report this, we would write “We reject the null hypothesis of no difference between mosquito abundances among the three locations (f(2,27) = 6.55, p = 0.0048, ANOVA)”. This one sentence conveys all the information we want to get across. Whenever you report the f

Post-hoc testing
Although ANOVA is wonderful and allows us to test multiple different groups, it doesn’t actually tell us which groups are different from each other. As this is useful and interesting, we need a way to do it. 

DON’T JUST DO A BUNCH OF T-TESTS
We cannot however just do a series of three pairwise t-tests, this is very naughty. With a t-test we’re looking to be 95% confident that there is a difference between the groups, we have a 5% chance of finding a difference when it isn’t there. Essentially, we’ve a 5% chance of being wrong. However, if we run three of tests together, we’re inflating the chance of finding at least one difference because due to probability theory our chances of being wrong at least once is…

1-(95%*95%*95%) = 1-85.7% = 14.3%

This means across our three tests, we have a 14.3% chance of finding at least one significant difference when it doesn’t exist. If we had 4 groups, and did all the pairwise t-tests (6 in total) our chance of being wrong increases to 26.5%, and it gets even worse every time you add another group. 

There is a solution, the Tukey test!
The Tukey test is a legitimate post-hoc test we can use to look for all the differences between our groups, it’s conservative, and very commonly used and accepted. In R, it’s also very easy to run, we just use the “TukeyHSD” command on whatever we named our ANOVA test, in our case, “anova.1”

TukeyHSD(anova.1)	\#\#\# Running a post-hoc Tukey test

This gives the following results table…

Tukey multiple comparisons of means
    95% family-wise confidence level

Fit: aov(formula = data$mosquitoes ~ data$location)

$`data$location`
                      diff        lwr        upr     p adj
Monte.verde-De.salva -45.7  -90.01399  -1.386014 0.0422571
pitilla-De.salva     -62.5 -106.81399 -18.186014 0.0045414
pitilla-Monte.verde  -16.8  -61.11399  27.513986 0.6202390

Beautiful, just look at it. It’s fitted our model (in the Fit: “aov….” Line) and then given us p-values for all of our location comparisons in the last column of the table. We can now just read these off and report them. We see Monte verde differed from De salva, pitilla differed from De salva, bit Pitilla didn’t differ from Monte verde. 





\end{document}
